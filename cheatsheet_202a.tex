% change font size here
\documentclass[12pt,landscape]{extarticle}
\usepackage{multicol}
\usepackage{calc}
\usepackage{ifthen}
\usepackage[landscape]{geometry}
\usepackage{amsmath,amsthm,amsfonts,amssymb}
\usepackage{color,graphicx,overpic}
\usepackage{hyperref}
\usepackage{bm}
\usepackage{mathtools}
\usepackage{dsfont}
\usepackage{pgfplots}
\usepackage{graphicx}
\usepackage[export]{adjustbox} % also loads graphicx


\pgfplotsset{width=7cm,compat=1.16}

\usepackage{tikz}
\usetikzlibrary{decorations.pathreplacing,calligraphy}
% define block style for automatons
\tikzstyle{block} = [rectangle, text centered, draw=black]
% arrow style
\tikzstyle{arrow} = [->,>=stealth]
\usetikzlibrary{calc}

\usetikzlibrary{decorations.pathmorphing}
\tikzset{snake/.style={decorate, decoration=snake}}


\pdfinfo{
  /Title (Econ202CheatSheet.pdf)
  /Creator (TeX)
  /Producer (pdfTeX 1.40.0)
  /Author (John Friedman)
  /Subject (ECON 202)
  /Keywords (pdflatex, latex,pdftex,tex)}

% This sets page margins to .5 inch if using letter paper, and to 1cm
% if using A4 paper. (This probably isn't strictly necessary.)
% If using another size paper, use default 1cm margins.
\ifthenelse{\lengthtest { \paperwidth = 11in}}
    { \geometry{top=.5in,left=.5in,right=.5in,bottom=.5in} }
    {\ifthenelse{ \lengthtest{ \paperwidth = 297mm}}
        {\geometry{top=1cm,left=1cm,right=1cm,bottom=1cm} }
        {\geometry{top=1cm,left=1cm,right=1cm,bottom=1cm} }
    }

% Turn off header and footer
\pagestyle{empty}

% Redefine section commands to use less space
\makeatletter
\renewcommand{\section}{\@startsection{section}{1}{0mm}%
                                {-1ex plus -.5ex minus -.2ex}%
                                {0.5ex plus .2ex}%x
                                {\normalfont\large\bfseries}}
\renewcommand{\subsection}{\@startsection{subsection}{2}{0mm}%
                                {-1explus -.5ex minus -.2ex}%
                                {0.5ex plus .2ex}%
                                {\normalfont\normalsize\bfseries}}
\renewcommand{\subsubsection}{\@startsection{subsubsection}{3}{0mm}%
                                {-1ex plus -.5ex minus -.2ex}%
                                {1ex plus .2ex}%
                                {\normalfont\small\bfseries}}
\makeatother

% Define BibTeX command
\def\BibTeX{{\rm B\kern-.05em{\sc i\kern-.025em b}\kern-.08em
    T\kern-.1667em\lower.7ex\hbox{E}\kern-.125emX}}

% Don't print section numbers
\setcounter{secnumdepth}{0}


\setlength{\parindent}{0pt}
\setlength{\parskip}{0pt plus 0.5ex}

%My Environments
\newtheorem{example}[section]{Example}
% -----------------------------------------------------------------------

\newcommand{\R}{\mathbb{R}}
\newcommand{\Rn}{\mathbb{R}^n}
\DeclareMathOperator*{\argmax}{arg max}
\DeclareMathOperator*{\argmin}{arg min}
\newcommand{\E}{\mathbb{E}}
\DeclareMathOperator{\unif}{unif}
\DeclareMathOperator{\co}{co}

\begin{document}
\raggedright
\footnotesize

\setlength{\columnsep}{0pt}
%Wait a few days. Go back thru comps from ECON department site, write down all social planner problems by hand, then check against notes.
%after that add interesting RCE
% add examples for log linearizing and balanced growth, transversality condition
% for any unique question, write down question with solution

% TODO look at original size, make sure not to make too big, e.g. add padding on side
%TODO  run over with llm to clena formmatting
% TODO eventually want lots of examples for each model, for now just get main points
\begin{multicols*}{3}
    \section{202A}

    \subsection{Social Planner's Problem} %TODO replace with examples
    \begin{enumerate}
        \item Write down all equations from the prompt
        \item Check if anything can be simplified easily. Do not simplify if it is not helpful
    \end{enumerate}

    \subsubsection{Tricks}
    Additive Consumption, e.g. 
    \subsubsection{Examples}

    \subsection{Recursive Competitive Equilibrium} %TODO, rewrite with examples
    A Recursive Competitive Equilibrium consists of:
    \begin{enumerate}
        \item Household's Problem
        \item Firm's Problem(s)
        \item Household decision rules
        \item Firm decision rules
        \item Pricing functions
        \item Aggregate Perception
        \item Pricing functions, Aggregate Perception, and Household decision rules solve Household's Problem
        \item Pricing function and Firm decision rules solve Firm's problem
        \item Market's clear
        \item Perception's are correct
    \end{enumerate}

    \subsection{Log Linearization} % https://drive.google.com/file/d/19z_Bcv7LBhtj6s1C_Gi2zybxa6P6-Ct2/view
    We define a variable in log deviation from its steady state as:
    $$ \hat{x} = \log(\frac{x}{\bar{x}})$$
    By reverse engineering, we can write $x$ as:
    $$x = \bar{x} \exp(\hat{x})$$
    The final step in the process, when the approximation plays a role, is that up to first order (meaning, $\hat{x}$ is small enough).
    $$\exp(\hat{x}) = f(0) + \sum_{i=1}^{\infty} \frac{\partial^i f}{\partial x^i}(0)\frac{(\hat{x}-0)^i}{i!} \approx 1 + \exp(0)(\hat{x}-0) = 1 + \hat{x}$$
    Then $x \approx \bar{x}(1+\hat{x})$
    \subsubsection{Tricks}
    Let $x,y,z$ be variables, and $\alpha, \beta, c$ constants.
    \begin{enumerate}
        \item $\hat{c} = 0$
        \item If $z = x + y$ then $\bar{z} \hat{z} \approx \bar{x} \hat{x} + \bar{y}\hat{y}$. And since $\bar{z} = \bar{x} + \bar{y}$, then
        $$\hat{z} \approx \frac{\bar{x}}{\bar{x}+{\bar{y}}} \hat{x} + \frac{\bar{y}}{\bar{x}+\bar{y}}\hat{y}$$
        \item if $z = x^\alpha y^\beta$ then
        $$\hat{z} = \alpha \hat{x} + \beta \hat{y}$$
    \end{enumerate}
    \subsection{Calibration}
    \subsection{Transversality Condition}
    \subsection{Characterization}
    Characterization is usually done using FOC, EC, and budget constraint

    \subsection{Unique Questions}
    \begin{enumerate}
        \item Derive expressions that determine how the planner allocates a given amount of capital and
        labor across the two market sectors. Prove that the same fraction of each input is allocated to
        a given sector in period t
        \item Show that the result obtained in part B can be used to aggregate the resource constraints (*)
        and (**) into one resource constraint (derive it). Repeat part A given this result.
        \item Suppose an empirical fact about the world is that as durable goods become cheaper,
        households spend less time in nonmarket activities (female labor supply increases, for
        example). If you wanted this model to be consistent with this fact, what would this mean for
        the functional form you would choose for the home production technology, F? (Note: This
        question is not so much asking for a functional form for F, although that would be fine, but is
        asking for list of properties that this function should possess.)
    \end{enumerate}



\end{multicols*}

\end{document}