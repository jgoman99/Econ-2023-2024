\documentclass[answers]{exam}

\usepackage{enumitem} % For customization of numbering
\usepackage{amsmath} % For \text
\usepackage [autostyle, english = american]{csquotes}
\MakeOuterQuote{"}

\title{201C HW1}
\author{John Friedman}
\date{\today}


\begin{document}
\maketitle

\begin{questions}
    \question Two firms ("agents") would like to invest in a city (e.g., real estate investments). The Government ("principal") is willing to let the agents invest in exchange for payments. Formally, the principal offers each agent $i \in \{1, 2\}$ a contract $\langle x_i, t_i \rangle$ which describes the investment level $x_i \geq 0$ and payment $t_i \geq 0$. The contracts are observed by both agents before they choose to accept.
    The principal obtains revenue $\pi = t_1 + t_2$. If agent $i$ accepts the contract, he makes utility $u_i = v_i - t_i$, where $v_i = x_i - \frac{1}{2} x_i^2 + \alpha x_j$. If agent $i$ rejects the contract, $\langle x_i, t_i \rangle = \langle 0, 0 \rangle$ and so he makes utility $u_i = v_i = \alpha x_j$. Note that $\alpha > 0$ means the agents have positive externalities on each other, while $\alpha < 0$ means the agents have negative externalities on each other. We assume that $\alpha > -1$.
    \begin{parts}
        \part What is the Pareto efficient investment level?
        \begin{solution}
            The Pareto efficient investment level is the level that maximizes the sum of the agents' and principal's utility level.
            \begin{align*}
                \max_{x_1, x_2} \quad t_1 + t_2 + u_1 + u_2\\
                \max_{x_1, x_2} \quad t_1 + t_2 + v_1 - t_1 + v_2 - t_2\\
                \max_{x_1, x_2} \quad v_1 + v_2\\
                \max_{x_1, x_2} \quad x_1 - \frac{1}{2} x_1^2 + \alpha x_2 + x_2 - \frac{1}{2} x_2^2 + \alpha x_1\\
                \text{FOC:}\\
                \frac{\partial}{\partial x_1} \quad 1 - x_1 + \alpha = 0\\
                \frac{\partial}{\partial x_2} \quad 1 - x_2 + \alpha = 0\\
                x_1 = 1 + \alpha\\
                x_2 = 1 + \alpha\\
                \end{align*}                
        \end{solution}

    \part Suppose the principal offers a "multilateral" contract $\langle x_i, t_i \rangle$ to each agent simultaneously. Formally:

    \begin{enumerate}
        \item[(i)]  The principal offers each agent a contract $\langle x_i, t_i \rangle$.
        \item[(ii)] Both agents choose to accept or reject.
        \item[(iii)] If either agent rejects, then both contracts are cancelled.
    \end{enumerate}
    
    What are the principal's optimal choices of $x_1$ and $x_2$?
    \begin{solution}
        The agent's choice is now between accepting a payoff of zero or $v_i-t_i$. Therefore the agent will accept any $t_i$ such that $v_i - t_i \geq 0$ and so the principal can extract the entire surplus. This has the same solution as the Pareto efficient investment level, as the goal of the principal is to maximize welfare since he can extract all of it. The optimal choices of $x_1$ and $x_2$ are $1 + \alpha$.
         
    \end{solution}
    \part \textbf{(c)} Suppose the principal offers a \emph{bilateral} contract $\langle x_i, t_i \rangle$ to each agent simultaneously. Formally:

    \begin{enumerate}
        \item[(i)]  The principal offers each agent a contract $\langle x_i, t_i \rangle$.
        \item[(ii)] Both agents choose to accept or reject.
        \item[(iii)] If an agent rejects, then the other agent's contract is unaffected.
    \end{enumerate}
    
    What are the principal's optimal choices of $x_1$ and $x_2$?
    \begin{solution}
        The agent's choice is now between accepting a payoff of $\alpha x_j$ or $x_i - \frac{1}{2} x_i^2 + \alpha x_j -t_i$. An agent should now accept any offer such that:
        \begin{align*}
            \alpha x_j \leq x_i - \frac{1}{2} x_i^2 + \alpha x_j -t_i\\
            0 \leq x_i - \frac{1}{2} x_i^2 -t_i\\
            \text{which, using FOC, is maximized at:}\\
            x_i = 1\\
        \end{align*}
        The principal's optimal choices of $x_1$ and $x_2$ are $1$.
    \end{solution}     
    \part How does the level of investment under multilateral and bilateral contracts depend on $\alpha$? Provide
    an intuition.
    \begin{solution} 
        Under multilateral contracts the level of investment will increase when $\alpha$ is positive and decrease when $\alpha$ is negative. The intuition here is that when externalities are positive, the principal can force a higher level of investment on both agents, as they will both benefit from the other's investment. When externalities are negative, the principal can't force as high of an investment level, as the agents will be hurt by the other's investment and will choose their outside option of 0 utility.
        \\\\
        Under bilateral contracts, the level of investment will always be 1. The intuition here is that bilateral negotiations allow the producer to extract the agent's production function, but not the impact of externalities on the agent, as the agent can always choose to reject the contract and recieve the externalities $\alpha x_j$.
        \end{solution}
    \part How does the principal's profit under multilateral and bilateral contracts depend on $\alpha$? Provide
    an intuition.
    \begin{solution}
        Under multilateral contracts, the principal's profit is maximized by maximizing surplus. If $\alpha \geq 0$, the principal extracts at least 1 unit of profit. However, if $\alpha < 0$, the principal's profit can be as low as zero.
        \\\\
        Under bilateral contracts, the principal can always extract the individual agent's maximized production $x_i - \frac{1}{2}x_i^2$ regardless of $\alpha$. This is great for the principal if $\alpha < 0$ as he can extract more than he would in the multilateral case, however in the case of $\alpha \geq 0$, the principal can't extract the whole surplus as in the multilateral case, as if he were to take more than $x_i-\frac{1}{2}x_i^2$, the agent would reject the contract and recieve $\alpha x_j$.
    \end{solution}
\end{parts}

\question And agent chooses effort $a \in A$ with increasing, convex cost $c(a)$. Suppose output $q \sim f(q|a)$ increases in effort in the sense of first-order stochastic dominance. The agent is risk averse with utility $u(w(q))-c(a)$, where $w(q)$ is their final compensation. The principal is risk-neutral with profits $q-w$. The agent has outside option $\bar{u}$, while the firm has all the bargaining power.
\begin{parts}
    \part Supose the principal can contract on the agent's effort. Characterize the first-best effort $\alpha^*$, expected output $q^*$, and expected profit, $\Pi^*$.
    \begin{solution} (Not sure if I need to show things bind)
        $a$ is contractible, so there is no moral hazard. The principal chooses the effort $a$ and the wage $w(q)$ to solve:
        \begin{align*} 
            \max_{a,w(q)} &E[q-w(q)|a]\\
            &s.t. \  (IR) \ E[u(w(q) - c(a))|a] \geq \bar{u}
        \end{align*}
        The "IR" stands for individual rationality. It means the agent is willing to sign the contract.
        \\\\
        We show that the (IR) binds in the optimal contract.
        \begin{itemize}
            \item If (IR) is slack, then we can lower the wage in each state. E.g. given contract $w(q)$ define a new contract $\tilde{w}(q) = w(q) - \epsilon$. 
            \item We will actually want a slightly stronger statement: That the Lagrange multiplier on the constraints is strictly positive
            \begin{itemize}
                \item This follows since a reduction in $\bar{u}$ allows the firm to lower wages and strictly increase profits.
            \end{itemize}
            Therefore the principal maximizes the Lagrangian:
            \begin{align*}
                \L = &E[q-w(q)|a] + \lambda (E[u(w(q) - c(a)  - \bar{u})|a])\\
                &= \int_{q} [q - w(q) + \lambda(q) + \lambda u (w(q))] d F[q|a] - \lambda c(a) - \lambda \bar{u}
            \end{align*}
            We show that the optimal contract fully insures the agent:
            \begin{itemize}
                \item What wages should the principal use? The above equation is additive over the states so we can maximize pointwise:
                $$\max_{w} -w +\lambda u(w)$$
                So the optimal choice of wage is independent of output. This objective is also concave.
                \item Using differentiation we find the principal sets $w(q) = w^* \ \forall q$ where $w^*$ is given by:
                $$\frac{1}{u'(w^*)} = \lambda$$
                \item $\lambda$ is the cost of delivering a util. Therefore the above equation shows that the cost of delivering a util is equal across all states and is given by $\lambda$.
                \end{itemize}
                We now find the optimal effort level.
                \begin{itemize}
                    \item the binding (IR) constraints show that:
                    $$u(w^*) -c(a) = \bar{u}$$
                    \item Therefore the principal chooses $a$ to maximize 
                    \begin{align*} 
                        \Pi &= E[q|a] - w^*\\
                        &= E[q|a] - u^{-1}(c(a)+\bar{u})
                    \end{align*}
                \end{itemize}
                The expected effort is $\Pi^* = q^* - w^*$ 
        \end{itemize}


    \end{solution}
    Now, suppose that both principal and agent can observe effort $a$, but can't contract on it. We wish to show that the following option contract implements the first-best. The agent first buys the firm for $\Pi^*$. After the effort is chosen, but before output is realized the principal then has the option to buy back the firm and obtain all the output for a price $q^*$.
    \part How does the principal's decision to exercise the option depend on the agent's choice of effort? 
    \begin{solution} (Better explanation needed?)
        If the agent does not choose the first-best effort, then $\Pi < q^*$ and the principal will choose not to buyback the firm. If the agent chooses the first-best effort, then $\Pi = q^*$ and the principal will choose to buyback the firm. The intuition here is that by selling the firm with a buyback option the principal pushes the cost of shirking onto the agent.
    \end{solution}
    \part Show that the agent will choose effort $a^*$ and recieve utility $\bar{u}$.
    \begin{solution}
        First note that this problem is equivalent to the one in lecture where the principal sells the firm and can't buy it back. This is because of the unstated assumption that the agent is risk-neutral, and the stated assumption that the principal can observe effort. Since the principal can observe effort, the principal will always sell the firm at the price of the first best effort, and so choosing to exercise the buyback in the first best effort case is equivalent to not buying the firm back in the first best effort.
    \\\\
    Suppose $u(w) = w$ the first best action $a^*$ maximizes:
    $$\Pi = E[q|a] - u^{-1}(c(a) + \bar{u}) = E[q|a] - c(a) -\bar{u}$$
    The agent's wage is $w(q) = q - \Pi^*$. Under this contract the agent chooses to maximize:
    $$U = E [u(w(q)- \Pi^* - c(a))] = E[u(q - \Pi^* - c(a))] = E[q|a] - \Pi^* - c(a)$$
    The optimal $\Pi^*$ makes the agent's (IR) bind:
    $$\Pi* = E[q|a^*] - c(a^*) - \bar{u}$$
    Therefore the agent chooses first best effort $a^*$ and recieves utility $\bar{u}$.
    \end{solution}


\end{parts}

\question The following normal-linear model is regularly used in applied models. Given action $a \in R$, output is $q=a+x$, where $x \sim N(0,\sigma^2)$. The cost of effort is $c(a)$ is increasing and convex. The agent's utility equals $u(w(q) - c(a))$, while the principal's is $q-w(q)$. Suppose the agent's outside option is $u(0)$.
\\\\
We make two large assumptions. First, the principal uses a linear contract:
$$w(q) = \alpha + \beta q$$
Second, the agent's utility is CARA, i.e. $u(w) = -e^{-w}$.

\begin{parts}
\part Suppose that $w \sim N(\mu, \sigma^2)$. Denote the certainty equivalent of $w$ by $\bar{w}$, where:
$$u(\bar{w}) = E[u(w)]$$
Show that $\bar{w} = \mu - \frac{\sigma^2}{2}$.
\begin{solution}
    Note: There is some messy math here that I don't want to write out. I ended up using Wikipedia's List of integrals of exponential functions to solve this.
    \begin{align*}
        u(\bar{w}) &= E[u(w)]\\
        -e^{-\bar{w}} &= E[-e^{-w}]\\
        -e^{-\bar{w}} &= \int_{-\infty}^{\infty} -e^{-w} \frac{1}{\sqrt{2 \pi}} e^{-\frac{1}{2}(\frac{w-\mu}{\sigma})^2}\\
        -e^{-\bar{w}} &= -e^{-\mu + \frac{\sigma^2}{2}}\\
        \bar{w} &= \mu - \frac{\sigma^2}{2}
    \end{align*}
\end{solution}
\part Suppose effort is unobservable. The principal's problem is:
\begin{align*}
    \max_{w(q), \alpha} E[q-w(q)]\\
    s.t. \ E[u(w(q) - c(a))| a] \geq u(0)\\
    a \in argmax_{a' \in R^+} E[(u(w(q))-c(a'))|a']
\end{align*}
Use the first order approach to implicitly characterize the optimal contract. $(\alpha, \beta, a)$ Hint: write utilities in terms of their certainty equivalent.
\begin{solution}
    TODO:
    First, note that this question is equivalent to the one presented on page 10 of the Moral Hazard lecture.
\end{solution}
\end{parts}

\question 

\end{questions}
    
\end{document}