\documentclass[answers]{exam}

\usepackage{enumitem} % For customization of numbering
\usepackage{amsmath} % For \text
\usepackage{hyperref}


\title{202A Question Bank}
\author{John Friedman}

\date{\today}
\hypersetup{
    colorlinks=true,   
    urlcolor=blue
    }

% Switch to doing hansen comps, then if need be do old finals

\begin{document}
\maketitle

\begin{questions}

    \question \href{https://economics.ucla.edu/wp-content/uploads/2016/10/macro-f16.pdf}{(Hansen Comp Spring 2016)}
    \href{https://drive.google.com/file/u/3/d/1I_XVSpJZWsAtp4uIZizpPaIZk6EHggCC/view?usp=drive_open}{(Solution)}
    (Similar to: Final 2017 Q1)

    $$E \sum_{t=0}^{\infty} \beta^t \{\frac{c_t^{1-\sigma}}{1-\sigma} - A \frac{h_t^{1+\frac{1}{\omega}}}{1+ \frac{1}{\omega}}\}, A> 0, \omega > 0, \sigma > 0$$
    where $0 < \beta < 1$. For each $t \geq 0$, the technology is given by:
    \begin{align*}
        &c_t + i_t = e^{z_t} (u_t k_t)^{\theta} h_t^{1-\theta}, \text{ where}\\
        &k_{t+1} = (1-\delta_t) k_t + i_t,\\
        &\delta_t = \delta u_t^\phi\\
        &z_{t+1} = \rho z_t + \epsilon_{t+1}, \text{where $\epsilon_{t}$ is iid with mean 0 and variance $\sigma_{\epsilon}^2$}
    \end{align*}
    The variable $u$ is a choice variable describing the fraction of the capital stock utilized in a given period, where $0 \leq u_t \leq 1$. The parameters $\phi > 1$ and $0 < \delta < 1$. Hours worked, $h$, is also constrained to be in the interval $[0,1]$.
    \begin{parts}
        \part Are the preferences of the representative agent consistent with balanced growth? Explain.
        \begin{solution}
            No. From Bangyu's Week 7 TA notes, for a model to exhibit balanced growth it must be characterized by one of the following utility functions:
            \begin{enumerate}
                \item $U(c,h) = c^{1-\sigma} v(l)$ for $\sigma \ne 1$
                \item $U(c,h) = \log{c} + v(l)$ for $\sigma = 1$
            \end{enumerate}
            Since the utility function above does not fit either of these forms, the preferences of the representative agent are not consistent with balanced growth.
            TODO: is this enough explanation?
        \end{solution}
        \part Carefully state the representative agent's dynamic programming problem for this economy.
        Obtain a set of equations that determine the optimal stochastic process for $\{k_{t+1}, c_t, h_t, u_t\}_{t=0}^\infty$.
        \begin{solution}
            The dynamic programming problem is:
            \begin{align*}
                V(k,z) &= \max_{c,h,u,k'} \frac{c^{1-\sigma}}{1-\sigma} - A \frac{h^{1+\frac{1}{\omega}}}{1+ \frac{1}{\omega}} + \beta E[V(k',z'|z,k)]\\
                &s.t.\ c + k' = e^{z} (u k)^{\theta} h^{1-\theta} + (1-\delta u^\phi) k\\
                &\text{and } z' = \rho z + \epsilon', \text{$\epsilon   '$ is iid with mean 0 and variance $\sigma_{\epsilon}^2$}
            \end{align*}
            The FOC are:
            \begin{align*}
                &[c]: c^{-\sigma} = \lambda\\
                &[h]: A h^{\frac{1}{\omega}} = \lambda e^{z} (1-\theta) (u k)^{\theta} h^{-\theta}\\
                &[u]:  \theta e^z u^{\theta-1}k^\theta h^{1-\theta} = \phi u^{\phi-1} k\\
                &[k']: \beta E[V_k(k',z'|z,k)] = \lambda
            \end{align*}
            The envelope condition is:
            $$
                E[V_k(k,z|z)] = \lambda (e^z \theta (u k)^{\theta-1} h^{1-\theta} + (1-\delta u^\phi))
            $$
            Iterate a period forward
            $$ E[V_k(k',z'|z)] = \lambda (e^{z'} \theta (u k')^{\theta-1} h^{1-\theta} + (1-\delta u^\phi))$$
            The Budget constraint is:
            $$c + k' = e^{z} (u k)^{\theta} h^{1-\theta} + (1-\delta u^\phi) k$$

            So we have for $c,h,u,k'$:
            \begin{align*} 
                c^{-\sigma} &= \beta E[V_k(k',z'|z,k)] = \lambda (e^{z'} \theta (u k')^{\theta-1} h^{1-\theta} + (1-\delta u^\phi))\\
                k' &= e^{z} (u k)^{\theta} h^{1-\theta} + (1-\delta u^\phi) k - c\\
                h^{\frac{\theta}{\omega}} &= \frac{1}{A}c^{-\sigma}(1-\theta) e^{z} (u k)^{\theta} \\
                u^{\theta-\phi} &= \frac{\phi \delta k^{1-\theta}}{\theta e^z h^{1-\theta}}
            \end{align*}

        \end{solution}
        \part Using the equations obtained in part B, determine if the rate of capital utilization is pro-cyclical or counter-cyclical. Explain [be specific by what you mean by pro (or counter)
        cyclical].
        \begin{solution}
            $u$ is pro-cyclical if it increases when $z$ increases. From the FOC for $u$:
            \begin{align*}
                u^{\theta-\phi} &= \frac{\phi \delta k^{1-\theta}}{\theta e^z h^{1-\theta}}\\
                u &= (\frac{\phi \delta}{\theta e^z})^\frac{1}{\theta-\phi} (\frac{k}{h})^\frac{1-\theta}{\theta-\phi}\\
            \end{align*}
            TODO: I'm skipping this for now, but essentially use this and FOCs to determine if $u$ is pro-cyclical or counter-cyclical.
        \end{solution}
        \part Define a recursive competitive equilibrium for this economy. Hint: Consider a market for
        utilized capital services rather than a capital rental market.
        \begin{solution}
            A recursive competitive equilibrium is defined by:
            \begin{enumerate}
                \item The household's problem:
                \begin{align*}
                    V(k,K,z) &= \max_{c,h,u,k'}  \frac{c^{1-\sigma}}{1-\sigma} - A \frac{h^{1+\frac{1}{\omega}}}{1+ \frac{1}{\omega}} \\
                    s.t. &\ c + k' = w(K,z)h + r(K,z)uk + (1-\delta u^\phi) k\\
                    &\text{and } z' = \rho z + \epsilon'\\
                    &\text{and } K' = G(K,z)
                \end{align*}
                \item The firm's problem:
                \begin{align*}
                    \max_{k^f,h^f} e^z k^{f \theta} h^{f 1-\theta} - r(K,z) {k}^f - w(K,z)h^f
                \end{align*}
                \item A set of household decisions rules: $c(k,K,z), h(k,K,z), u(k,K,z), k'(k,K,z)$
                \item A set of firm decision rules: $k^f(K,z), h^f(K,z)$
                \item A set of pricing functions $r(K,z), w(K,z)$ where consumption price is the numeraire
                \item A law of motion for the state variables $K' = G(K,z)$ such that:
                \begin{enumerate}
                    \item Given pricing functions and perceived law of motion for aggregate capital, the household decision rules solve the household's problem
                    \item Giving pricing functions, firm decision rules solve the firm's problem.
                    \item Markets clear:
                    \begin{enumerate}
                        \item $h^f(K,z) = h(k,K,z)$
                        \item $\tilde{u}^f(K,z) = u(k,K,z)$
                        \item consumption market clears by walras's law
                    \end{enumerate}
                    \item There are rational expectations: $G(K) = k'(K,K,z)$
                \end{enumerate}
            \end{enumerate}
        \end{solution}
        \part The Frisch elasticity of labor supply is defined to be the elasticity of hours worked to the
        wage rate holding the marginal utility of wealth (the Lagrange multiplier for the budget
        constraint) constant. Derive the Frisch elasticity for your decentralized economy. Make sure
        you clearly explain your derivation.
        \begin{solution}
            First, we use logs on the FOC for $h$, and substitute in for the wage term holding $\lambda$ constant:
            \begin{align*} 
                A h^{\frac{1}{\omega}} = \lambda e^{z} (1-\theta) (u k)^{\theta} h^{-\theta}\\
                A h^{\frac{1}{\omega}} = \lambda w\\
                \log{A} + \frac{1}{\omega} \log{h} = \log{\lambda} + \log{w}\\
            \end{align*}
            Second, we take the derivative of the log of hours worked wrt to the log of the wage rate:
            \begin{align*}
                \frac{d \log{h}}{d \log{w}} = \frac{d h}{d w} = \omega
            \end{align*}
            Therefore, the Frisch elasticity of labor supply is $\omega$.
        \end{solution}
        \part Derive a log-linear approximation that expresses the percentage deviation of hours worked as
        a function of the percentage deviation of the wage rate and consumption from their steady
        states. Assuming consumption is equal to its steady state, use this equation to determine how
        large is the standard deviation of hours divided by the standard deviation for the wage rate in
        this economy.
        \begin{solution}
            Using FOC for $h$ and $c$:
            \begin{align*}
                c^{-\sigma} &= \lambda\\
                A h^{\frac{1}{\omega}} &= \lambda (w) \text{ where $w$ is the wage rate}\\
                \implies w c^{-\sigma} &= A h^{\frac{1}{\omega}}\\
                \tilde{w} - \sigma \tilde{c} &= \frac{1}{\omega} \tilde{h} \text{ using the log linearization shortcut Bangyu taught us}\\
                \omega &= \frac{\tilde{h}}{\tilde{w}} \text{ since consumption is equal to its steady state, the deviation of consumption is 0}
            \end{align*}

        \end{solution}
        \part Discuss the implications of this model for using the Solow residual or total factor
        productivity to measure exogenous technical progress.
        \begin{solution}
            We normally have: $F(k,h) = e^z k^\theta h^{1-\theta}$. But now we have $F(k,h,u) = e^z (uk)^\theta h^{1-\theta}$.


            Normally, we need meaures of total output, capital, and hours worked to estimate the Solow residual. However, in this case we also need measures of $u$ the fraction of capital stock utilized in a given period.
        \end{solution}
    \end{parts}        
    \question \href{https://economics.ucla.edu/wp-content/uploads/2016/10/macro-f16.pdf}{(Hansen Comp Fall 2016)}
    Consider a real business cycles model with a representative household that lives forever and maximizes the following utility function:
    $$E_0 \sum_{t=0}^{\infty} \beta^t \{ \log{c_t} + A \log{L_t} \}, 0 < \beta < 1, A > 0$$
    Here, $c_t$ is consumption and $L_t$ is a convex combination of leisure in periods $t$ and $t-1$. Each period, households are assumed to have one unit of time that can be allocated between market work, $h_t$, and leisure. In particular, let $L_t = a(1-h_t) + (1-a)(1-h_{t-1})$ where $0 \leq a \leq 1$. 


    Output, which can be used for consumption, investment ($i_t$) or government purchases is produced according to a constant returns to scale technology, $y_t = e^{z_t} k_t^\theta h_t^{1-\theta}$, where $y_t$ is output and $k_t$ is the stock of capital. The variable $z_t$ is a technology shock observed at 
    the beginning of period $t$ that evolves through time according to a first order autoregressive process with mean
    zero innovations. The stock of capital is assumed to depreciate at the rate $\delta$ each period. 


    Investment in period $t$ becomes productive capital one period later, $k_{t+1} = (1-\delta)k_t + i_t$. Government spending is an exogenous random variable that, like the technology shock, follows a first order autoregressive process, in this case with an unconditional mean $\bar{g}$ and unconditional variance $\sigma_g^2$. Innovations to this process are assumed 
    to be independent of innovations to the technology shock process and the value of $g_t$ is observed at the beginning of period $t$. In addition, government purchases are financed with lump sum taxes. Note that government purchases do not directly affect preferences or the technology; they are simply thrown into the sea.
    \begin{parts}
        \part Are the equilibrium allocations for this economy the solution to a social planner’s problem?
        Explain. If so, write the social planners problem for this economy as a dynamic
        programming problem. Be specific about the stochastic process (law of motion) for $z_t$ and $g_t$.
        \part Derive as a set of equations that characterize a sequence $\{c_t,h_t,L_t,k_{t+1},y_t\}_{t=0}^\infty$ that solves the social planner’s problem. Be sure that you have the same number of equations as unknowns. Explain the role of the transversality condition in determining this optimal sequence.
        \part Define a recursive competitive equilibrium for this economy. 
        \part Assume that for a given variable $x, \tilde{x_t} \sim \log{x_t} - \log{\bar{x}}$ where $\bar{x}$ is the non-stochastic steady state value of $x_t$. Derive a linear expression for $\tilde{h_t}$ as a function of $\tilde{k_t}, \tilde{c_t}, z_t, \tilde{h_{t-1}}$.
        \part As in the previous part, derive a linear equation expressing $\tilde{c_t}$ as a function of $\tilde{k_t}, z_t, \tilde{g_t}, \tilde{h_t}, \tilde{k_{t+1}}$.
        \part In a standard real business cycle model, $a=1$. Using the equation derived in part (d and or e), explain how setting $a<1$ might change the cyclical properties of the model economy. In particular, focus on the size of fluctuations in hours worked relative to $z_t$. Provide intuition in your explanation.
        \part In a standard real business cycle model, $g_t$ is not included as a stochastic shock. Discuss
        how adding this feature might change the cyclical properties of the model economy. In
        particular, focus on the correlation between hours worked and $z_t$. Again, provide intuition.
    \end{parts}
    \question \href{https://drive.google.com/drive/folders/1UWwY7p_JSQYuK9Ux_qW2YtpXHbVJyX4T}{(Hansen Final 2017 Q1)} \href{https://drive.google.com/file/d/1mlICFZK5ci4eUi5QhH7ZqUfNSqtUGcTc/view}{(Solution)} Consider a stochastic growth model where a representative household's preferences are given by,
    $$E \sum_{t=0}^{\infty} \beta^t {\log{c_t} - A \frac{h_t^{1+\frac{1}{\omega}}}{1+ \frac{1}{\omega}}}, A> 0, \omega > 0$$
    where $0 < \beta < 1$. For each $t \geq 0$, the technology is given by:
    \begin{align*}
        c_t + i_t &= e^{z_t} (u_t k_t)^{\theta} h_t^{1-\theta}, \text{ where}\\
        k_{t+1} &= (1-\delta_t) k_t + i_t,\\
        \delta_t &= \delta u_t^\phi\\
        z_{t+1} &= \rho z_t + \epsilon_{t+1}, \text{where $\epsilon_{t}$ is iid with mean 0 and variance $\sigma_{\epsilon}^2$}
    \end{align*}
    The variable $u$ is a choice variable describing the fraction of the capital stock utilized in a given period, where $0 \leq u_t \leq 1$. The parameters $\phi > 1$ and $0 < \delta < 1$. Hours worked, $h$, is also constrained to be in the interval $[0,1]$.
    \begin{parts}
        \part Carefully state the representative agent's dynamic programming problem for this economy. Obtain a set of equations that, along with a transversality condition, determine the optimal stochastic process for $\{k_{t+1}, c_t, h_t, u_t, \delta_t\}_{t=0}^\infty$.
        \begin{solution}
            The dynamic programming problem is:
            \begin{align*}
                V(k,z) &= \max_{c,h,u,k'} \log{c} - A \frac{h^{1+\frac{1}{\omega}}}{1+ \frac{1}{\omega}} + \beta E[V(k',z'|z)]\\
                &s.t.\ c + k' = e^{z} (u k)^{\theta} h^{1-\theta} + (1-\delta u^\phi) k,\\
                &\text{and } z_{t+1} = \rho z_t + \epsilon_{t+1}
            \end{align*}
            We can simplify the problem by taking the FOCs wrt to $c,h,u$:
            \begin{align*}
                &[c]: \frac{1}{c}  = \lambda\\
                &[h]: A h^{\frac{1}{\omega}} = \lambda e^{z} (1-\theta) (u k)^{\theta} h^{-\theta}\\
                &[u]:  \theta e^z u^{\theta-1}k^\theta h^{1-\theta} = \phi u^{\phi-1} k\\
                &[k']: \beta E[V_k(k',z'|z)] = \lambda
            \end{align*}
            TODO: envelope condition: Not sure? transversality. fix subscritps, fix E. note policy function and market for capital ask bangyu
        \end{solution}
        \part Define a recursive competitive equilibrium for this economy. Hint: consider a market for utilized capital services rather than a capital rental market.
        \begin{solution}
            A recursive competitive equilibrium is defined by:
            \begin{enumerate}
                \item The household's problem:
                \begin{align*}
                    V(k,z) &= \max_{c,h,u,k'} \log{c} - A \frac{h^{1+\frac{1}{\omega}}}{1+ \frac{1}{\omega}} \\
                    s.t. &\ c + k' = w(K,z)h + r(K,z)\tilde{k} + (1-\delta u^\phi) k\\
                    &\text{and } z_{t+1} = \rho z_t + \epsilon_{t+1}\\
                    &\text{and } \tilde{u} = uk\\
                    &\text{and } K' = G(K,z)
                \end{align*}
                \item The firm's problem:
                \begin{align*}
                    \max_{\tilde{u}^f,h^f} e^z \tilde{u}^{f \theta} h^{f 1-\theta} - r(K,z) \tilde{h}^f - w(K,z)h^f
                \end{align*}
                \item A set of household decisions rules: $c(k,K,z), h(k,K,z), u(k,K,z), k'(k,K,z)$
                \item A set of firm decision rules: $\tilde{u}^f(K,z), h^f(K,z)$
                \item A set of pricing functions $r(K,z), w(K,z)$ where consumption price is the numeraire
                \item A law of motion for the state variables $K' = G(K,z)$ such that:
                \begin{enumerate}
                    \item Given pricing functions and perceived law of motion for aggregate capital, the household decision rules solve the household's problem
                    \item Giving pricing functions, firm decision rules solve the firm's problem.
                    \item Markets clear:
                    \begin{enumerate}
                        \item $h^f(K,z) = h(k,K,z)$
                        \item $\tilde{u}^f(K,z) = u(k,K,z)$
                        \item consumption market clears by walras's law
                    \end{enumerate}
                    \item There are rational expectations: $G(K) = k'(K,K,z)$
                \end{enumerate}
            \end{enumerate}
        \end{solution}
        \part The Frish elasticity of labor supply is defined to be the elasticity of hours worked to the wage rate holding the marginal utility of wealth (The Lagrange multiplier for the budget constraint) constant. Derive the Frisch Elasticity for your decentralized economy. Make sure you clearly explain your derivation.
        \begin{solution}
            First, we use logs on the FOC for $h$, and substitute in for the wage term holding $\lambda$ constant:
            \begin{align*} 
                A h^{\frac{1}{\omega}} = \lambda e^{z} (1-\theta) (u k)^{\theta} h^{-\theta}\\
                A h^{\frac{1}{\omega}} = \lambda w\\
                \log{A} + \frac{1}{\omega} \log{h} = \log{\lambda} + \log{w}\\
            \end{align*}
            Second, we take the derivative of the log of hours worked wrt to the log of the wage rate:
            \begin{align*}
                \frac{d \log{h}}{d \log{w}} = \omega
            \end{align*}
            Therefore, the Frisch elasticity of labor supply is $\omega$.
        \end{solution}
        \part Derive a log-linear approximation that expresses the percentage-deviation of hours worked as a function of the percentage deviation of the wage rate and consumption from their steady states. Assuming consumption is equal to its steady state, use this equation to determine how large is the standard deviation of hours divided by the standard deviation for the wage rate in this economy. TODo hint bangyu week 9a. add full way and bangyu way
        \begin{solution}
            Using FOC for $h$ and $c$:
            \begin{align*}
                A h^{\frac{1}{\omega}} = \frac{w}{c}\\
                \implies A h^{\frac{1}{\omega}} c = w\\
                \leftrightarrow A (\bar{h} e^{-\tilde{h}})^{\frac{1}{\omega}} (\bar{c} + e^{-\tilde{c}}) = \bar{w} e^{-\tilde{w}}\\
                \leftrightarrow A (\bar{h}^\frac{1}{\omega} e^{-\frac{\tilde{h}}{\omega}}) (\bar{c} e^{-\tilde{c}}) = \bar{w} e^{-\tilde{w}}\\
                \leftrightarrow A (\bar{h}^\frac{1}{\omega} (1 + \frac{\tilde{h}}{\omega})) (\bar{c} (1 + \tilde{c})) = \bar{w} (1 + \tilde{w})\\
            \end{align*}
        \end{solution}
        \part Discuss the implications of this model for using the Solow residual or total factor productivity to measure exogenous technical progress.
        \begin{solution}
            TODO. I don't remember if we discussed this in class.
        \end{solution}
    \end{parts}
    % is c) for that question only? We have calculated everything else based on that assumption.
    \question \href{https://drive.google.com/file/d/1Oc2RsQ-J3MtMMVoAwpuAdPodoOrIJZmm/view}{(Final 2016 Q1)}
    Consider an economy population by a continuum of identical households with the following preferences:
    $$\sum_{t=0}^{\infty} \beta^t \ln(c_t + A \ln l_t), 0 < \beta < 1, $$
    where $c_t$ is consumption and $l_t$ is leisure at date $t$. Households are endowed with one unit of time each period that can be used for labor or leisure. In addition, each household is endowed with $k_0$ units of capital in period 0 and can accumulate capital according to the law of motion:
    $$k_{t+1} = (1-\delta) k_t + i_t, 0 < \delta < 1$$
    where $i_t$ is investment at date $t$.\\
    The households sell labor to a competitive firm and can work either a straight time shift of length $h_1$, a straight time plus overtime shift of length $h_1+h_2$, or not at all (thus, labor is an indivisible commodity). The technology for combining capital with straight time and overtime labor to produce output $(y_t)$ is given by:
    $$y_t = e^{z_t}(h_1 k_t^\theta (n_{1t}+n_{2t})^{(1-\theta)} + h_2 k_t^\theta n_{2t}^{(1-\theta)}), 0 < \theta < 1$$
    where $n_{1t}$ is the number of households working only straight time and $n_{2t}$ is the number of households working both straight time and overtime. Output can be used for current consumption or investment. The technology shock, $z_t$ evolves according to:
    $$z_{t+1} = \rho z_t + \epsilon_{t+1}, \epsilon_{t+1} \text{ is iid random variable with mean 0}$$
    \begin{parts}
        \part Carefully formulate the dynamic program that would be solved by a social planner that chooses capital, labor, and consumption sequences to maximize a social welfare function that weights all agent's utilities equally. 
        \begin{solution}
            The dynamic programming problem is:
            \begin{align*}
                V(k,z) &= \max_{c_1,c_2,c_3, n_1, n_2, k'}  n_1 \ln{(c_1 + A \ln(1-h_1))} + n_2 \ln{(c_2 + A \ln(1-h_1- h_2))} \\
                &+ (1-n_1 -n_2) \ln(c_3) + \beta E[V(k',z'|z)]\\
                &s.t.\ n_1 c_1 + n_2 c_2 + (1-n_1-n_2) c_3 + k' - (1-\delta) k = e^{z}(h_1 k^\theta (n_{1}+n_{2})^{(1-\theta)} + h_2 k^\theta n_{2}^{(1-\theta)})\\
                &\text{and } z_{t+1} = \rho z_t + \epsilon_{t+1}, \epsilon_{t+1} \text{ is iid random variable with mean 0}
            \end{align*}
            We can simplify the problem further by proving seperability of consumption. Taking the FOCs wrt to consumption:
            \begin{align*}
                [c_1]: \frac{n_1}{c_1} - \lambda = 0\\
                [c_2]: \frac{n_2}{c_2} - \lambda = 0\\
                [c_3]: \frac{1-n_1-n_2}{c_3} - \lambda = 0
            \end{align*}
            We can see from the FOCs that the consumption is seperable. Therefore, we can simplify the dynamic programming problem to:
            \begin{align*}
                V(k,z) &= \max_{c, n_1, n_2, k'}  \ln(c) +  n_1 \ln{( A \ln(1-h_1))} + n_2 \ln{(A \ln(1-h_1- h_2))} \\
                &+ \beta E[V(k',z'|z)]\\
                &s.t.\ c + k' - (1-\delta) k = e^{z}(h_1 k^\theta (n_{1}+n_{2})^{(1-\theta)} + h_2 k^\theta n_{2}^{(1-\theta)})\\
                &\text{and } z_{t+1} = \rho z_t + \epsilon_{t+1}, \epsilon_{t+1} \text{ is iid random variable with mean 0}
            \end{align*}
        \end{solution}
        \part Prove that in equilibrium the fraction of employed households that work overtime is constant even when the economy is not in a steady state
        \part Suppose that there are moving costs that must be incurred when the number of straight time workers is changed, $m_t = \frac{d}{2}(n_{1t} - n_{1t-1})^2$. The output available for consumption and investment is, in this case $y_t - m_t$.  Repeat A for this case and show that the statement in part B no longer holds.
        % note that n_1{t=0} is state variable. n1 determines n2, and we cant use n2 as state variable due to moing costs
        \part Define a recursive competitive equilibrium for the model of part (A) where agents trade employment lotteries. Be sure to completely specify the problem solved by households and firms in your decentralized economy.
        \part Derive an expression for the straight time hourly wage rate and the overtime wage rate as a function of the prices determined in part D. Next, derive an expression for the overtime wage premium, which is the ratio of the hourly overtime wage rate to the hourly straight-time wage rate in terms of the parameters of the model. Under what conditions will the overtime premium be greater than one?
    \end{parts}

    \question \href{https://drive.google.com/file/d/1Oc2RsQ-J3MtMMVoAwpuAdPodoOrIJZmm/view}{(Final 2016 Q2)}
    Consider a two-sector economy with technological progress and population groqth. The first sector combines capital and labor to produce a consumption good according to the resource constraint:
    $$C_t = \gamma_1^t K_{1,t}^\theta H_{1,t}^{1-\theta}, \gamma > 1$$
    The second sector uses capital and labor to produce new capital and a consumer durable good. In particular,
    $$\frac{N_{t+1}}{N_t}(D_{t+1} + K_{t+1}) = \gamma_2^t K_{2,t}^\theta H_{2,t}^{1-\theta}+ (1 - \delta_k) K_t + (1-\delta_D)D_t, \gamma_2 > 1$$
    Here all variables are in per capita units. $K_t = K_{1,t} + K_{2,t}$ is physical capital, $D_t$ is the stock of consumer durables, and $H_1 = H_{1,t} + H_{2,t}$ is hours worked. Assume that the population $N_t$ evolves according to the law of motion $N_{t+1} = \eta N_t, \eta > 1$.  Note that productivity grows in the two sectors but at potentially different rates.
    \\
    Assume that an infinitely lived representative household values nondurable consumption, durables and leisure according to the period utility function,
    $$\alpha \log C_t + (1-\alpha) \log D_t + \phi log (1-H_t)$$
    Households maximize the sum of utility at each date with future utility discounted at the rate $0 < \beta < 1$. 
    \begin{parts} %TODO, check if mobility of capital effects anything, e.g. can we use growth rate of sector 1 capital and 2, or just groth rate of capital
        \part Find a change of variables so that the resource constraints in terms of the transformed variables are stationary. That is, they do not depend on calendar time (t)
        \begin{solution}
            First, we assume that hours worked are constant. (TODO)
        \end{solution}
        \part Formulate a stationary dynamic programming problem solved by a social planner who puts equal weight on all individuals
        \part Derive expressions that determine how the planner allocates a given amount of capital and labor across the two sectors. Prove that the same fraction of each input is allocated to a given sector in period $t$. That is, show that $h_{1,t} = \phi_t H_t$ and $K_{1,t} = \theta_t K_t$. Obviously the remainder, a fraction $1-\phi_t$, is allocated to sector 2.
        \part Show that the result obtained in part C can be used to aggregate the sectoral resource constraints into one resource constraint (derive it).
        \part Characterize the balanced growth path for this economy
        \part Explain how the equations you obtained in part C can be used to calibrate the parameters $\alpha, \beta, \phi, \delta_K, \delta_D$. Be explicit about what sorts of data would be required and what statistics you would need to calculate from the data (to save time you don't need to go into detail about HOW you would calculate these statistics).
    \end{parts}
\question \href{https://drive.google.com/file/d/1Oc2RsQ-J3MtMMVoAwpuAdPodoOrIJZmm/view}{(Final 2016 Q3)}
\begin{parts}
    \part Suppose that $z_t$ can take on one of three possible values, {$a_1,a_2,a_3$}. Suppose that the stochastic process for the shock $z_t$ is a Markov chain with transition matrix P.  (check all)
    \begin{enumerate}
        \item Provide an example of a transtion matrix P such that there is a unique invariant distribution and one transient state. Describe how to find the invariant distribution.
        \begin{solution}
            Remember that an item in the transition matrix $P$, $P_{ij}$ is the probability of moving from state $i$ to state $j$. A transient state is a state where it is possible to leave and never return. An invariant distribution is a distribution that is unchanged by the transition matrix. Such a matrix is:
            $$P = \begin{bmatrix}
                0 & .5 & .5\\
                0 & .5 & .5\\
                0 & .5 & .5
            \end{bmatrix}$$
        \end{solution}
        \item Provide an example of a transition matrix P such that there are three ergodic sets. Characterize the invariant distribution(s) in this case. Given an initial distribution, $\pi_0$, what will be the limit distribution?
        \begin{solution}
            An ergodic set is a set of states, which once reached, will never leave. An example of such a matrix is:
            $$P = \begin{bmatrix}
                0 & 1 & 0\\
                0 & 0 & 1\\
                1 & 0 & 0
            \end{bmatrix}$$
            TODO: given an initial distribution what will be the limit distribution:
        \end{solution}
        \item Provide an example of a transition matrix P such that the invariant distribution is unique and places equal probability on all three states.
        \begin{solution} (CHECK, better explanation)
            $$P = \begin{bmatrix}
                q & \frac{1-q}{2} & \frac{1-q}{2}\\
                \frac{1-q}{2} & \frac{1-q}{2} & 1\\
                \frac{1-q}{2} & \frac{1-q}{2} & q
            \end{bmatrix}$$
        \end{solution}
        \item Consider now a 2-state Markov chain. Suppose you are told that the unconditional mean of z is zero, the unconditional variance of z is equal to $\sigma^2$, and the first order autocovariance of z is equal to b. what values for $\{a_1,a_2\}$ and the transition matrix P would be consistent with these restrictions?
        \item Suppose we solve a standard stochastic growth where the capital stock must lie on a grid and the transition matrix is your answer to part a of this question. That is, $k_t \in \{k_1,k_2,...,k_M\}$ for all $t$ and we know the optimal law of motion $K' = G(z,K)$. Describe how this can be used to solve for the invariant distribution for this model. How can one determine if the invariant distribution is unique?
    \end{enumerate}
\end{parts}

\question \href{https://drive.google.com/drive/folders/11xTRC6EWE1VcPfk5BbXZRyGUX5oQ-jvo}{(Comp 2017)} Consider an economy with a representative household with $N_t$ identical members. The household's preferences are given by,
$$\sum_{t=0}^{\infty} \beta^t N_t [\log{c_t} + A log(1-h_t)]$$
Each member of the household is endowed with one unit of labor each period. The number of members evolves over time according to the law of motion, $N_{t+1} = \eta N_t, \eta>1$. \\
Output is produced using the following technology:
$$Y_t = \gamma^t e^{z_t} K_t^\mu (N_t h_t)^{\phi} L_t^{1-\mu - \phi}$$

Here $\gamma > 1$ is the gross rate of exogeneous total factor productivity growth, $K_t$ is the total (not per capita) capital, $Y_t$ is the total output, and $L_t$ is the total stock of land. Land is assumed to be a fixed factor; it cannot be produced and does not depreciate. To simplify without loss of generality, assume that $L_t = 1$ for all $t$. \\
The variable $z_t$ is a technology shock that follows an autoregeressive process, $z_{t+1} =\rho z_t + \epsilon_{t+1}$, where $\epsilon$ is independently and identically distributed over time with mean 0 and standard deviation $\sigma$.\\
The resource constraint, assuming 100 percent depreciation of capital each period, is given by:
$$N_t c_t + K_{t+1} = Y_t$$
\begin{parts}
    \part Formulate the social planning problem for this economy as a stationary dynamic program. Be clear about the transformation performed such that all the variables are stationary.
    \begin{solution}
        \begin{enumerate}
            \item Normalize the population so $N_0 = 1 \implies N_{t} = \eta^t$.
            \item Simplify the total capital equation $K_t = N_t k_t = \eta^t k_t$
            \item Substitute for $Y_t$ in the resource constraint: $N_t c_t + K_{t+1} = \gamma^t e^{z_t} K_t^\mu (N_t h_t)^{\phi} L_t^{1-\mu - \phi}$
            \item Substitute $N_t$ and $K_t$: $\eta^t c_t + \eta^{t+1} k_{t+1} = \gamma^t e^{z_t} (\eta^t k_t)^\mu (\eta_t h_t)^{\phi} L_t^{1-\mu - \phi}$
            \item Since $L_t=1$: $K_t$: $\eta^t c_t + \eta^{t+1} k_{t+1} = \gamma^t e^{z_t} (\eta^t k_t)^\mu (\eta_t h_t)^{\phi}$
            \item (IMPROVE) Define the following terms: $c_t = g_c^t \hat{c}_t,\ k_t = g_k^t \hat{k}_t,\ h_t = g_h^t \hat{h}_t$
            \item (improve) $g_h = 1$ as we assume hours worked is constant 
            \item Substitute into the resource constraint: $\eta^t g_c^t \hat{c}_t + \eta^{t+1} g_k^{t+1} \hat{k}_{t+1} = \gamma^t e^{z_t} (\eta^t g_k^t \hat{k}_t)^\mu (\eta_t g_h^t \hat{h}_t)^{\phi}$
            \item Group terms by $t$: $(\eta g_c)^t \hat{c}_t+ (\eta g_k)^{t+1} \hat{k}_{t+1} =  e^{z_t} \hat{k}_t^\mu \hat{h}_t^\phi (\gamma \eta^{\mu + \phi} g_k^\mu)^t$
            \item Divide by $g_c^t$ and $\eta^t$: $\hat{c}_t + \eta g_k (\frac{g_k}{g_c})^t \hat{k}_{t+1} = e^{z_t} \hat{k}_t^\mu \hat{h}_t^\phi (\frac{\gamma \eta^{\mu + \phi - 1} g_k^\mu}{g_c})^t$
            \item Therefore, to be stationary, $g_k=g_c$ and $g_c = \gamma \eta^{\mu + \phi - 1} g_k^\mu$
            \item Using $g_k=g_c \implies g_k=g_c = (\gamma \eta^{\mu + \phi -1})^{\frac{1}{1-\mu}}$ 
            \item Substitute $g_k$ into the resource constraint: $\hat{c}_t + \eta (\gamma \eta^{\mu + \phi -1})^{\frac{1}{1-\mu}} (1)^t \hat{k}_{t+1} = e^{z_t} \hat{k}_t^\mu \hat{h}_t^\phi (1)^t$
            \item We do the same for preferences 
            \begin{align*}
                \sum_{t=0}^{\infty} \beta^t N_t [\log{c_t} + A log(1-h_t)]\\
                \sum_{t=0}^{\infty} \beta^t \eta^t [\log{(g_c^t \hat{c}_t)} + A \log(1-g_h^t \hat{h}_t)]\\
                \sum_{t=0}^{\infty} \beta^t \eta^t [\log{(g_c^t \hat{c}_t)} + A \log(1- \hat{h}_t)]\\
            \end{align*}
            \item The social planning problem is now:
            \begin{align*}
                \max_{\hat{c}_t,\hat{h}_t,\hat{k}_{t+1}}\sum_{t=0}^{\infty} \beta^t \eta^t [\log{(g_c^t \hat{c}_t)} + A \log(1- \hat{h}_t)]\\
                \text{s.t. } \hat{c}_t + \eta (\gamma \eta^{\mu + \phi -1})^{\frac{1}{1-\mu}} (1)^t \hat{k}_{t+1} = e^{z_t} \hat{k}_t^\mu \hat{h}_t^\phi (1)^t\\
                \text{and } z_{t+1} = \rho z_t + \epsilon_{t+1} \text{, where $\epsilon \sim N(0,\sigma^2)$ and $\epsilon$ is iid}\\
                \text{given $k_0, z_0$}
            \end{align*}
            \item Or we can use the Dynamic Programming problem:
            \begin{align*}
                V(\hat{k},z) = \max_{\hat{c},\hat{h},\hat{k}'} \log{\hat{c} + A \log(1-\hat{h})} + \beta \eta E[V(\hat{k}',z'|z)]\\
                \text{s.t. } \hat{c} + \eta (\gamma \eta^{\mu + \phi -1})^{\frac{1}{1-\mu}} \hat{k}' = e^{z} \hat{k}^\mu \hat{h}^\phi\\
                \text{and } z' = \rho z + \epsilon \text{, where $\epsilon \sim N(0,\sigma^2)$ and $\epsilon$ is iid}
            \end{align*}
        \end{enumerate}

        \end{solution}
    \part Characterize the balanced growth path of this economy. That is, find expressions that determine $c_t$
    $h_t$ and $K_t$ along this growth path. In particular, solve explicitly for the growth rate of this set of
    variables.
    \begin{solution}
        \begin{enumerate}
            \item To characterize the balanced growth path of this economy we take the FOC: (IMPROVE: why can we drop subscripts)
            \begin{align}
                [c]: \frac{1}{\hat{c}} - \lambda = 0\\
                [h]: -\frac{A}{1-\hat{h}} + \lambda \phi e^z \hat{k}^{\mu} \hat{h}^{\phi-1} = 0 \\
                [k]: \mu e^{z} \hat{k}^{\mu-1} \hat{h}^\phi = \eta (\gamma \eta^{\mu + \phi -1})^{\frac{1}{1-\mu}} \text{ Using the envelope condition }
            \end{align}
            \item We then combine (1) with (2), use the fact that in steady state hat variables are constant, and add in the resource constraint for the steady state:
            \begin{align} 
                [h]: -\frac{A}{1-\bar{h}} + \frac{1}{\bar{c}} \phi e^z \bar{k}^{\mu} \bar{h}^{\phi-1} = 0 \\
                [k]: \mu e^{z} \bar{k}^{\mu-1} \bar{h}^\phi = \eta (\gamma \eta^{\mu + \phi -1})^{\frac{1}{1-\mu}}\\
                \bar{c}_t + \eta (\gamma \eta^{\mu + \phi -1})^{\frac{1}{1-\mu}} \bar{k}_{t+1} = e^{z_t} \bar{k}_t^\mu \bar{h}_t^\phi
            \end{align}
        \item These three equations characterize the balanced growth path for $\bar{c},\bar{h},\bar{k}$ that we would then plug in for:
        \begin{align}
            c_t = g_c^t \bar{c}\\
            h_t = g_h^t \bar{h}\\
            K_t = g_k^t \eta^t \bar{k}
        \end{align}
        \item We found the explicit growth rate of these variables in the previous part:
            $$g_k=g_c = (\gamma \eta^{\mu + \phi -1})^{\frac{1}{1-\mu}}, g_h = 1$$
        \end{enumerate}
    \end{solution}
    \part Discuss how your answer to part B would change if $\phi = 1 - \mu$. In particular, what is the growth rate
    of income per capita in the two cases (part A and B)?
    \begin{solution} (IMPROVE)
        We still have $c_t = g_c^t \bar{c}, h_t = g_h^t \bar{h}, K_t = g_k^t \eta^t \bar{k}$, but now $g_k=g_c=\gamma^\frac{1}{1-\mu}$. And our charateristic equations are now:
        \begin{align} 
            [h]: -\frac{A}{1-\bar{h}} + \frac{1}{\bar{c}} \phi e^z \bar{k}^{\mu} \bar{h}^{\phi-1} = 0 \\
            [k]: \mu e^{z} \bar{k}^{\mu-1} \bar{h}^\phi = \eta (\gamma \eta^{\mu + \phi -1})^{\frac{1}{1-\mu}}\\
            \bar{c}_t + \eta (\gamma \eta^{\mu + \phi -1})^{\frac{1}{1-\mu}} \bar{k}_{t+1} = e^{z_t} \bar{k}_t^\mu \bar{h}_t^\phi
        \end{align}
        To find the growth rate per capita of income, let's first simplify the income equation:
        \begin{align*} 
            Y_t = \gamma^t e^{z_t} K_t^\mu (N_t h_t)^{\phi} L_t^{1-\mu - \phi}\\
            \eta^t y_t = \gamma^t e^{z_t} (\eta^t k_t)^\mu (\eta^t h_t)^{1-\mu} (1)\\
            y_t = \gamma^t e^{z_t} k_t^\mu h_t^{1-\mu}
        \end{align*}
        The change in income between periods in the steady state is 
        $$\frac{y_{t+1}}{y_t} = \gamma \frac{e^{z_{t+1}} k_{t+1}^\mu h_{t+1}^{1-\mu}}{e^{z_t} k_t^\mu h_t^{1-\mu}}$$
        Since we are in steady state we can simplify capital, leisure, and the technology shock:
        $$\frac{y_{t+1}}{y_t} = \gamma$$
        Therefore the per capita growth rate of income is $y^t$
    \end{solution}
    \part Suppose that a period is one quarter and suppose you are given annual growth rates for population
    and per capita income. You are also given values for factor shares, the average amount of time spent
    working and the annual capital-output ratio. Show how these facts can be used to calibrate $\gamma,\eta,A, \beta$
    \begin{solution}
        \begin{enumerate}
            \item Since $\text{annual growth rate} = (\text{quarterly growth rate})^4 \\ \implies \eta = (\text{annual growth rate of population})^\frac{1}{4}$
            \item TODO
        \end{enumerate}
    \end{solution}
    \part Define a recursive competitive equilibrium for this economy assuming markets for labor,
    consumption goods, land rental, and capital services.
    \begin{solution}
        Note: that we drop the hats. The household problem is:
        \begin{align*}
            V(k,K,z) = max_{c,h,k',l} \log{c} + A \log(1-h) + \beta \eta E[V(k',K',z'|z)]\\
            s.t.\ c + \eta (\gamma \eta^{\mu + \phi -1})^{\frac{1}{1-\mu}} k' = hw(K,z) + kr(K,z) + t(K,z)l\\
            K' = G(K)\\
            z_{t+1} = \rho z_t + \epsilon_{t+1}, \text{where $\epsilon \sim N(0,\sigma^2)$ and $\epsilon$ is iid}\\
            \text{given $k_0, z_0$}
        \end{align*}
        \begin{align*}
            \text{The firm problem is:}\\
            \max_{h^f,l^f,k^f}[e^{z_t}(h^f)_t^\phi (k_f)_t^\mu l_f^{1-\mu-\phi} - t(K,z)l^f - k^f r(K,z) - h^f w(K,z)]
        \end{align*}
        A recursive competetive equilibrium is defined by:
        \begin{itemize}
            \item A set of household decision rules $c(k,K,z), h(k,K,z), k'(k,K,z), l(k,K,z)$
            \item A set of firm decision rules $h^f(K,z), l^f(K,z), k^f(K,z)$
            \item A set of pricing functions $w(K,z), r(K,z), t(K,z)$ where consumption price is the numeraire
            \item Perceived law of motion for aggregate capital $G(K)$ such that:
            \begin{itemize}
                \item Given pricing functions and a percieved law of motion for aggregate capital, the household decision rules solve the household problem
                \item Given pricing functions, firm decision rules solve the firm's problem
                \item Markets clear
                \begin{itemize}
                    \item $h^f(K,z) = h(K,K,z)$ 
                    \item $l^f(K,z) = l(K,K,z)$
                    \item $k^f(K,z) = K$
                    \item The consumption market by Walras' Law
                \end{itemize}
                Note that little $k$ is now big $K$, as the firm optimizes over aggregate capital (improve)
                \item Rational expectations: $G(K) = k'(K,K,z)$
            \end{itemize}
        \end{itemize}
    \end{solution}
    \part Add a real estate market to your equilibrium definition in part E. Derive an equation determining the
    price of land.
    \begin{solution}
        Note: that we drop the hats. The household problem is:
        \begin{align*}
            V(k,K,l,L,z) = max_{c,h,k',l'} \log{c} + A \log(1-h) + \beta \eta E[V(k',K',l',L',z'|z)]\\
            s.t.\ c + \eta (\gamma \eta^{\mu + \phi -1})^{\frac{1}{1-\mu}} k' + s(K,z,L)l' = hw(K,z,L) + kr(K,z,L) + (t(K,z,L)+s(K,z,L))l\\
            K' = G(K)\\
            L' = H(L)\\
            z_{t+1} = \rho z_t + \epsilon_{t+1}, \text{where $\epsilon \sim N(0,\sigma^2)$ and $\epsilon$ is iid}\\
            \text{given $k_0, z_0$}
        \end{align*}
        \begin{align*}
            \text{The firm problem is:}\\
            \max_{h^f,l^f,k^f}[e^{z_t}(h^f)_t^\phi (k_f)_t^\mu l_f^{1-\mu-\phi} - t(K,z,L)l^f - k^f r(K,z,L) - h^f w(K,z,L)]\\
        \end{align*}
        A recursive competetive equilibrium is defined by:
        \begin{itemize}
            \item A set of household decision rules $c(k,K,z,l,L), h(k,K,z,l,L), k'(k,K,z,l,L), l(k,K,z,l,L)$
            \item A set of firm decision rules $h^f(K,z,L), l^f(K,z,L), k^f(K,z,L)$
            \item A set of pricing functions $w(K,z,L), r(K,z,L), t(K,z,L)$ where consumption price is the numeraire
            \item Perceived law of motion for aggregate capital $G(K)$ and land $H(L)$ such that:
            \begin{itemize}
                \item Given pricing functions and a percieved law of motion for aggregate capital and land, the household decision rules solve the household problem
                \item Given pricing functions, firm decision rules solve the firm's problem
                \item Markets clear
                \begin{itemize}
                    \item $h^f(K,z,L) = h(K,K,L,L,z)$ 
                    \item $l'(K,z,L) = l'(K,K,L,L,z)$
                    \item $l^f(K,z,L) = l(K,K,L,L,z)$
                    \item $k^f(K,z,L) = K$
                    \item The consumption market by Walras' Law
                \end{itemize}
                \item Rational expectations: $G(K) = k'(K,K,z,L,L)$ and $H(L) = l'(K,K,z,L,L)$
            \end{itemize}
        \end{itemize}
    \end{solution}
\end{parts}
\end{questions}
\end{document}