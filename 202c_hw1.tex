\documentclass[answers]{exam}
\renewcommand\partlabel{\arabic{partno}.}
\renewcommand\questionlabel{\textbf{Question \arabic{question}:}}

\usepackage{enumitem} % For customization of numbering
\usepackage{amsmath} % For \text
\usepackage [autostyle, english = american]{csquotes}
\MakeOuterQuote{"}

\title{202C HW1}
\author{John Friedman}
\date{\today}

\begin{document}
\maketitle

\begin{questions}
    \question In this question, we consider a 4-country version of the Armington model. Consider the following parameter values: $\sigma = 3$, $a_{i,j} = 2$ for all $i = j$, $a_{i,j} = 1$ for all $i \neq j$, $L_1 = 2$, $L_2 = L_3 = L_4 = 1$, $A_1 = A_2 = A_3 = A_4 = 0.6$, and trade cost $\tau_{i,j}$ corresponds to the element $i, j$ in the following matrix:
    $$ T = \begin{pmatrix}
        1 & 1.1 & 1.2 & 1.3 \\
        1.3 & 1 & 1.3 & 1.4 \\
        1.2 & 1.2 & 1 & 1.1 \\
        1.1 & 1.1 & 1.1 & 1
      \end{pmatrix}
      $$      
      \begin{parts}
      \part Solve for equilibrium wages in countries 2, 3 and 4 relative to country 1.
     \begin{solution}
        \begin{enumerate}
            \item We will use the good of country 1 as the numeraire ($p_1 =1)$ 
            \item The excess demand function of the Armington Model is: 
            $$Z_i(w) = \{\sum_{j \in S} (\frac{a_{ij}\tau_{ij}^{1-\sigma}(\frac{w_i}{A_i})^{1-\sigma}}{\sum_{k \in S} a_{kj} \tau_{kj}^{1-\sigma}(\frac{w_k}{A_k})^{1-\sigma}})\frac{w_j L_j}{w_i}\}- L_i$$
            \item We solve by guessing a vector of wages, and iterating until the excess demand function is zero.
            
            Using the code attached in the appendix:
            $\frac{w_2}{w_1} = 1.1458, \frac{w_3}{w_1} = 1.2238, \frac{w_4}{w_1} = 1.2614$
        \end{enumerate}
     \end{solution} 
     \part Solve for bilateral trade shares, $\lambda_{ij}$ for $i = 1, 2, 3, 4$ and $j = 1, 2, 3, 4$
     \begin{solution}
        Since we have already solved for wages, we can use the formula $\lambda_{ij} = \frac{X_{ij}}{Y_j}$ To solve for bilateral trade shares we use the code attached in the appendix. We get:
        $$\lambda = \begin{pmatrix}
        0.5825 & 0.248 & 0.2315 & 0.2122 \\
        0.1313 & 0.4571 & 0.1502 & 0.1393 \\
        0.135 & 0.1391 & 0.4451 & 0.1978 \\
        0.1513 & 0.1558 & 0.1731 & 0.4507 \\
        \end{pmatrix}
        $$
     \end{solution}
     \part Consider (only in this question) that country 2's productivity increases by a factor of 2, from $A_2 = 0.6$ to $A_2' = 1.2$, while the others remain unchanged.
            \begin{enumerate}[label=\alph*)]
                \item What's the change in welfare for country 2 from the productivity shock?
                \begin{solution}
                    The formula for welfare is $W_i = \lambda_{ii}^{\frac{1}{1-\sigma}} a_{ii}^{\frac{1}{\sigma - 1}} A_i$ So using our code:
                    $$W_i^{new}/W_i^{old} = \frac{2.2863}{1.2551 } = 1.82$$
                \end{solution}
                \item What’s the change in welfare for country 2 from the productivity shock under
                autarky ? (hint: you simply need to use an equation from the lecture and no need
                to solve the model)
                \begin{solution}
                    From Sunny's first section, we know that 
                    \begin{align*}
                        W_i = \lambda_{ii}^{\frac{1}{1-\sigma}} a_{ii}^{\frac{1}{\sigma-1}} A_i \\
                        \text{The change in welfare is:}\\
                        \frac{W_i^{new}}{W_i^{old}} = \frac{\lambda_{ii}^{\frac{1}{1-\sigma}} \alpha_{ii}^{\frac{1}{\sigma-1}} A_i'}{\lambda_{ii}^{\frac{1}{1-\sigma}} \alpha_{ii}^{\frac{1}{\sigma-1}} A_i} = \frac{A_i'}{A_i} = \frac{1.2}{0.6} = 2
                    \end{align*}
                \end{solution}
                \item Provide intuition for the difference in your answers in (a) and (b)
                \begin{solution}
                    In part (a) and (b), we showed that the increase in welfare from a home productivity shock is bigger under autarky than in an open economy for the affected country.
                    \\
                    The intuition for this is that in an open economy, as country 2 becomes more productive it increases the supply of goods to the rest of the world, which pushes down its relative price and import share. e.g. Country 2's terms of trade decrease. In autarky, the country fully absorbs the benefits of its productivity shock, as it is not trading with the rest of the world.
                \end{solution}
            \end{enumerate}
        \part     $$ T = \begin{pmatrix}
            1 & 1 & 1.2 & 1.2 \\
            1 & 1 & 1.2 & 1.2 \\
            1 & 1.2 & 1 & 1.3 \\
            1 & 1.2 & 1.2 & 1
          \end{pmatrix}
          $$     
          \begin{enumerate}
            \item Solve for equilibrium wages in countries 2, 3 and 4 relative to country 1.
            \begin{solution} 
                $\frac{w_2}{w_1} = 1.2337 \frac{w_3}{w_1} =1.2356 \frac{w_4}{w_1} = 1.2506$
                
            \end{solution}
            \item Solve for bilateral trade shares, $\lambda_{ij}$ for $i = 1, 2, 3, 4$ and $j = 1, 2, 3, 4$
            \begin{solution}
                $$\lambda = \begin{pmatrix}
                \begin{pmatrix}
                0.5062 & 0.3112 & 0.2391 & 0.2465 \\
                0.1663 & 0.409 & 0.1571 & 0.162 \\
                0.1658 & 0.1416 & 0.451 & 0.1376 \\
                0.1618 & 0.1382 & 0.1528 & 0.4539 \\
                \end{pmatrix}
                \end{pmatrix}
                $$
              \end{solution}
          \end{enumerate}
          \part Solve for the change in wage in each country (relative to country 1’s wage) using the
          system in changes discussed in Section 5 in the Lecture Notes. Verify that you get the
          same result as in 4. What is the advantage of solving the system in changes rather
          than in levels (two times)?
          \begin{solution}
            The hat-algebra produces the same result. The advantage of solving the system in changes is that it requires less data: Elasticity of substitution, consumption shares, output shares, and iceberg costs in both cases.

          \end{solution}

    \end{parts}
    \question In this question, we consider a specification of the armington model in which each country produces output 
    using labor and intermediate inputs (rather than just labor). Intermediate inputs are made of the same final good as consumption. That it, intermediate outputs contain the same import content as consumption.\\
    Specifically, output in country $i$ is produced according to:
    $$Q_i = A_i L_i^\alpha M_i^{1-\alpha}$$
    where $L_i$ denotes labor (in fixed supply) and $M_i$ denotes the use of intermediate input. In our baseline model, we have assumed $\alpha = 1$.\\
    The intermediate input is made of the same final good that is used for consumption (welfare). Specifically, we assume that a final good is produced in each country $j$ according to the argmington aggregator: $(\sum_{i\in S} a_{ij}^{\frac{1}{\sigma}} q_{ij}^\frac{\sigma-1}{\sigma})^{\frac{\sigma}{\sigma - 1}}$. Given this technology, 
    competitive final good firms purchase individual goods $q_{ij}$ to produce the final good.
    Profit maximization
    by these final good producers results in the standard CES demand that we derived in the
    Armington model. The final good is then used for production ($M_j$) and for consumption
    by households ($C_j$). Households derive utility from consumption of the final good, $u(C_j)$
    Therefore, the resource constraint for the final good is given by:
    $$(\sum_{i\in S} a_{ij}^{\frac{1}{\sigma}} q_{ij}^\frac{\sigma-1}{\sigma})^{\frac{\sigma}{\sigma - 1}} = M_j + C_j$$
    Note that, in our baseline model, $M_j = 0$.\\ 
    The resource cosntraint for output produced in country $i$ is:
    $$Q_i = \sum_{j} \tau_{ij} q_{ij}$$
    \begin{parts}
        \part Show that the output price at the factory door, $p_i$, is given by:
        \begin{align*}
            p_i &= \kappa \frac{w_i^\alpha P_i^{1-\alpha}}{A_i} \\
            \text{where } \kappa &= \frac{1}{\alpha^\alpha (1-\alpha)^{1-\alpha}}\\
            \text{and } P_j &= (\sum_{i'\in S} a_{i'j} p_{i' j}^{1-\sigma} )^{\frac{1}{1-\sigma}} \\
            \text{and } p_{ij} &= \tau_{ij} p_j
        \end{align*}
        \begin{solution}
            Using cost minimization for the firm that produces a variety $i$ is:
            $$min_{L_i,M_i} w_i L_i + P_i M_i \text{ s.t. } A_i L_i^\alpha M_i^{1-\alpha} \geq \bar{Q_i}$$
            The Lagrangian is:
            $$\mathcal{L} = w_i L_i + P_i M_i - \lambda_i (\bar{Q_i} - A_i L_i^\alpha M_i^{1-\alpha})$$
            FOC:
            \begin{align}
                [L_i]: & w_i - \lambda_i \alpha A_i Li^{\alpha-1} M_i^{1-\alpha} = 0\\
                [M_i]: & P_i - \lambda_i (1-\alpha) A_i Li^{\alpha} M_i^{-\alpha} = 0\\
                [\lambda_i]: & A_i L_i^\alpha M_i^{1-\alpha} - \bar{Q_i} = 0
            \end{align}
            From the FOC wrt to labor:
            $$L_i = (\frac{\alpha \lambda_i A_i}{w_i})^{\frac{1}{1-\alpha}} M_i$$
            We plug into the FOC wrt to intermediate inputs:
            $$P_i = \lambda_i (1-\alpha) A_i (\frac{\alpha \lambda_i A_i}{w_i})^{\frac{\alpha}{1-\alpha}} $$
            $$P_i = \lambda_i ^{\frac{1}{1-\alpha}} A_i^{\frac{1}{1-\alpha}} (1-\alpha) (\frac{\alpha}{w_i})^{\frac{\alpha}{1-\alpha}}$$
            Note that $\lambda_i$ is our marginal cost for good from country $i$:
            $$\lambda_i = A_i^{-1} \frac{P_i}{1-\alpha}^{1-\alpha} (\frac{w_i}{\alpha})^\alpha$$
            In equilibrium, the marginal cost should be equal to the price $p_i = \frac{1}{A_i} (\frac{P_i}{1-\alpha})^{1-\alpha}(\frac{w_i}{\alpha})^\alpha$. Then $p_{ij} = \tau_{ij} p_i$.
            $$p_{ij} = \tau_{ij} \frac{1}{A_i} (\frac{P_i}{1-\alpha})^{1-\alpha}(\frac{w_i}{\alpha})^\alpha =  \kappa \frac{\tau_{ij}w_i^\alpha P_i^{1-\alpha}}{A_i}$$
            We now define a final good producer that uses intermediate varietes:
            $$Y_j = (\sum_{i\in S} a_{ij}^{\frac{1}{\sigma}} q_{ij}^\frac{\sigma-1}{\sigma})^{\frac{\sigma}{\sigma - 1}} P_j$$
            The optimal demand is
            $$q_ij = a_{ij}(\frac{p_{ij}}{P_j})^{-\sigma} \frac{Y_j}{P_j}$$
            Where $P_j = (\sum_{i\in S} a_{ij} p_{ij}^{1-\sigma})$
        \end{solution}
        \part Show that the value of use of the intermediate inputs and the wage bill in country $i$ must satisfy:
        \begin{align}
             P_i M_i &= (1-\alpha)p_i Q_i\\
             w_i L_i &= \alpha p_i Q_i
        \end{align}
        \begin{solution}
            $$\max{L_i,M_i} p_i A_i L_i^\alpha M_i^{1-\alpha} - w_i L_i - p_i M_i$$
            \begin{align*}
                \text{[$w_i$]} &= \alpha p_i A_i L^{\alpha-1} M_i^{1-\alpha} - w_i = 0\\
                w_i &= \alpha p_i A_i L^{\alpha-1} M_i^{1-\alpha}\\
                w_i L_i &= \alpha p_i Q_i\\ 
                \text{[$M_i$]} &= (1-\alpha) p_i A_i L_i^\alpha M_i^{-\alpha} - p_i = 0\\
                p_i &= (1-\alpha) p_i A_i L_i^\alpha M_i^{-\alpha}\\
                P_i M_i &= (1-\alpha)p_i Q_i
            \end{align*}
        \end{solution}
        where $p_i$ is the "factory door" price (e.g. without shipping costs) of the good produced in country $i$, and $P_i$ is the price of the final good (and henced intermediate inputs).
        \part The general equilibrium conditions of the model are now given by:
        \begin{align}
            P_i C_i &= w_i L_i \text{ where } w_i L_i \text{ is GDP in country } i, \\
             p_i Q_i &= \sum_{j \in S} X_{ij}\\
        \end{align}
            Equation (3) is the budget constraint indicating the consumption expenditures must be equal to income. Equation (4) indicates that the value of production must be equal to total sales. Show that:
            $$ P_i (C_i + M_i) = \frac{1}{\alpha} w_i L_i$$
        \begin{solution}
            From the previous part we know that $P_i M_i = (1-\alpha)p_i Q_i$ and $w_i L_i = \alpha p_i Q_i$. 
            \begin{align*}
                P_i M_i &= (1-\alpha)(\frac{w_i L_i}{\alpha})\\
                P_i M_i + (P_i C_i) &= \frac{1-\alpha}{\alpha} w_i L_i + (w_i L_i)\\
                P_i (M_i + C_i) &= \frac{1}{\alpha} w_i L_i
            \end{align*}
        \end{solution}
        \part Show that the ratio of sales from country $i$ to $j$ relative to country $j$'s GDP is
        $$\lambda_{ij} = \frac{X_{ij}}{w_j L_j} = a_{ij} \tau_{ij}^{1-\sigma} (\kappa \frac{w_i^\alpha P_i^{1-\alpha}}{A_i})^{1-\sigma} \frac{1}{\alpha} P_j^{\sigma-1}$$
        and that the ratio of sales from country $i$ relative to $i$'s GDP is
        $$\lambda_{ii} = \frac{a_{ii}}{\alpha} (\frac{\kappa}{A_i})^{1-\sigma} (\frac{w_i}{P_i})^{\alpha(1-\sigma)}$$
        \begin{solution}
            
        \end{solution}
        \part How do the gains from trade (for a given change in $\lambda_{ii}$) depend on $\alpha$? Provide intuition for your answer.
        \part How does the answer to the previous question change if the intermediate input in each country is made of domestic labor rather than made of the final consumption good.
    \end{parts}
    \question In this question, we consider a 2-country Armington model which does not impose that countries have balanced trade every period. Specifically, set $\sigma = 3$, $a_{ij} = 1 \ \forall i,j$. $L_1,L_2 = 1$, $A_1 = A_2 = 0.6$ and trade costs $\tau_{ij}$ correspond to the element $i,j$ in the following matrix:
    $$T = \begin{pmatrix}
        1 & 1.3 \\
        1.3 & 1
        \end{pmatrix}$$
    Different from what we have done in class, now we suppose that country $i$ income is given by:
    $$Y_i = w_i L_i + D_i$$
    where $D_i \ne 0$ is expressed in terms of the world numeraire. The vector $[D_1, D_2]$ is exogenously given and satisfies $D_1 + D_2 = 0$. For the numerical example below assume $D = [0.05, -0.05]$.
    \begin{parts}
        \part Show that country $i$'s trade deficit (imports minus exports) is equal to $D_i$
        $$D_i = \sum_{j \ne i} X_{ji} - \sum_{j \ne i} X_{ij} = D_i$$
        \begin{solution}
            \begin{align*}
                Y_i &= X_{ji} + X_{ii} \text{ income is equal to imports}\\
                D_i + w_i L_i &= X_{ji} + X_{ii} \text{ Using formula for income}\\
                D_i + X_{ij} + X_{ii} &= X_{ji} + X_{ii} \text{ Wage income is equal to value of goods consumed domestically and abroad}\\
                D_i &= X_{ij} - X_{ji}
            \end{align*}
        \end{solution}
        \part Characterize the equilibrium conditions to solve for the relative wages across countries. Solve for $w_2$ (normalizing $w_1 = 1$) in the numerical example.
        \begin{solution}
            We know that price at factory board is equal to marginal cost of production, $p_i = \frac{w_i}{A_i}$. Since we know $w_1=1 \implies p_1 = \frac{5}{3}$

            TODO. Finish. Note: I don't get how iceberg costs work in context of equations. Do they delete goods?
        \end{solution}
        \part Suppose that the world moves to financial autarky, so that countries must run a trade balance $D_1 = D_2 = 0$. What are the new equilibrium wages? Calculate changes in welfare for each country. Provide intuition for your answer.\
        \begin{solution}
            Unless I am missing something, the new equilibrium wages are meaningless as they are dependent on the price level in their country.
            $$p_i = \frac{w_i}{A_i}, W_i = \frac{w_i}{P_i}, \text{ and } P_i = p_i \text{ due to no trade}$$
            $$\implies W_i = A_i = .6 \text { for all countries}$$
            TODO: add welfare changes
        \end{solution}
    \end{parts}
    \end{questions}
\end{document}